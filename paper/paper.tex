% Options for packages loaded elsewhere
\PassOptionsToPackage{unicode}{hyperref}
\PassOptionsToPackage{hyphens}{url}
%
\documentclass[
]{article}
\title{histoc: Histocompatibility analysis performed by kidney
allocation systems}
\author{}
\date{\vspace{-2.5em}24 July 2022}

\usepackage{amsmath,amssymb}
\usepackage{lmodern}
\usepackage{iftex}
\ifPDFTeX
  \usepackage[T1]{fontenc}
  \usepackage[utf8]{inputenc}
  \usepackage{textcomp} % provide euro and other symbols
\else % if luatex or xetex
  \usepackage{unicode-math}
  \defaultfontfeatures{Scale=MatchLowercase}
  \defaultfontfeatures[\rmfamily]{Ligatures=TeX,Scale=1}
\fi
% Use upquote if available, for straight quotes in verbatim environments
\IfFileExists{upquote.sty}{\usepackage{upquote}}{}
\IfFileExists{microtype.sty}{% use microtype if available
  \usepackage[]{microtype}
  \UseMicrotypeSet[protrusion]{basicmath} % disable protrusion for tt fonts
}{}
\makeatletter
\@ifundefined{KOMAClassName}{% if non-KOMA class
  \IfFileExists{parskip.sty}{%
    \usepackage{parskip}
  }{% else
    \setlength{\parindent}{0pt}
    \setlength{\parskip}{6pt plus 2pt minus 1pt}}
}{% if KOMA class
  \KOMAoptions{parskip=half}}
\makeatother
\usepackage{xcolor}
\IfFileExists{xurl.sty}{\usepackage{xurl}}{} % add URL line breaks if available
\IfFileExists{bookmark.sty}{\usepackage{bookmark}}{\usepackage{hyperref}}
\hypersetup{
  pdftitle={histoc: Histocompatibility analysis performed by kidney allocation systems},
  hidelinks,
  pdfcreator={LaTeX via pandoc}}
\urlstyle{same} % disable monospaced font for URLs
\usepackage[margin=1in]{geometry}
\usepackage{graphicx}
\makeatletter
\def\maxwidth{\ifdim\Gin@nat@width>\linewidth\linewidth\else\Gin@nat@width\fi}
\def\maxheight{\ifdim\Gin@nat@height>\textheight\textheight\else\Gin@nat@height\fi}
\makeatother
% Scale images if necessary, so that they will not overflow the page
% margins by default, and it is still possible to overwrite the defaults
% using explicit options in \includegraphics[width, height, ...]{}
\setkeys{Gin}{width=\maxwidth,height=\maxheight,keepaspectratio}
% Set default figure placement to htbp
\makeatletter
\def\fps@figure{htbp}
\makeatother
\setlength{\emergencystretch}{3em} % prevent overfull lines
\providecommand{\tightlist}{%
  \setlength{\itemsep}{0pt}\setlength{\parskip}{0pt}}
\setcounter{secnumdepth}{-\maxdimen} % remove section numbering
\newlength{\cslhangindent}
\setlength{\cslhangindent}{1.5em}
\newlength{\csllabelwidth}
\setlength{\csllabelwidth}{3em}
\newlength{\cslentryspacingunit} % times entry-spacing
\setlength{\cslentryspacingunit}{\parskip}
\newenvironment{CSLReferences}[2] % #1 hanging-ident, #2 entry spacing
 {% don't indent paragraphs
  \setlength{\parindent}{0pt}
  % turn on hanging indent if param 1 is 1
  \ifodd #1
  \let\oldpar\par
  \def\par{\hangindent=\cslhangindent\oldpar}
  \fi
  % set entry spacing
  \setlength{\parskip}{#2\cslentryspacingunit}
 }%
 {}
\usepackage{calc}
\newcommand{\CSLBlock}[1]{#1\hfill\break}
\newcommand{\CSLLeftMargin}[1]{\parbox[t]{\csllabelwidth}{#1}}
\newcommand{\CSLRightInline}[1]{\parbox[t]{\linewidth - \csllabelwidth}{#1}\break}
\newcommand{\CSLIndent}[1]{\hspace{\cslhangindent}#1}
\ifLuaTeX
  \usepackage{selnolig}  % disable illegal ligatures
\fi

\begin{document}
\maketitle

\hypertarget{summary}{%
\section{Summary}\label{summary}}

The distribution of a scarce commodity such as deceased donor's kidneys
for transplantation should be as equitably as possible. Different
countries try to implement kidney allocation systems (KAS) in
transplantation that balance principles of justice and utility in the
distribution of such scarce resource
(\protect\hyperlink{ref-Lima:2020}{Lima and Alves 2020}). That is, a KAS
should optimize the transplant clinical outcome (principle of utility)
while giving a reasonable opportunity to all wait list candidates to be
transplanted (principle of justice)
(\protect\hyperlink{ref-Geddes:2005}{Geddes et al. 2005}).

The selection of a donor-recipient pair in kidney transplantation is
based on histocompatibility tests that can eliminate specific transplant
candidates from opting for a kidney from a given deceased donor. These
histocompatibility tests are used in several KAS and can be specific to
each of them.

The goal of this package is to aid the evaluation and assessment of KAS
in transplantation.

\hypertarget{statement-of-need}{%
\section{Statement of need}\label{statement-of-need}}

\{histoc\} (\protect\hyperlink{ref-histoc}{Lima and Reis 2022}) is an R
(\protect\hyperlink{ref-R}{R Core Team 2021}) package that assembles
tools for histocompatibility testing in the context of kidney
transplantation. The package's main functions allow simulating several
KAS on the distribution of deceased donors' grafts for transplantation.
Moreover, it is possible to redefine arguments for each one of the KAS
as a way to test different approaches.

Currently, it is possible to simulate allocation rules implemented in
Portugal (PT model), in countries within Eurotransplant (ET model)
(\protect\hyperlink{ref-ET}{EuroTransplant 2020}), in the United Kingdom
(UK model) (\protect\hyperlink{ref-UK}{Blood and Transplant 2017}), and
a system suggested by \protect\hyperlink{ref-Lima:2013}{Lima, Mendes,
and Alves} (\protect\hyperlink{ref-Lima:2013}{2013}) (Lima's model).

Each one of these models have as arguments a data frame with transplant
candidates' clinical and demographic characteristics, a data frame with
candidates' HLA antibodies and potential donor's information.

For all the models a virtual crossmatch between the donor and transplant
candidates is performed (\texttt{xmatch()}). And, only those candidates
with a negative crossmatch and ABO compatible can opt to a donor's
kidney.

Results are presented as \{data.table\}
(\protect\hyperlink{ref-data.table}{Dowle and Srinivasan 2021}) objects
due to its high computation performance.

To get started a vignette describes
\href{https://txopen.github.io/histoc/articles/how_to.html}{how to use}
each of the algorithms.

By default 2 candidates are selected for each donor, although we can
define the number of candidates to be selected.

New kidney allocation systems should be assessed using simulations that,
to the greatest extent possible, can predict findings and outcomes.
\{histoc\} is designed mainly for researchers working on organ
transplantation to aid decision making based on data, regarding the
definition of allocation policies.

While the R package \{transplantr\}
(\protect\hyperlink{ref-transplantr}{Asher 2020}) make available a set
of functions for audit and clinical research in transplantation, the
package presented here enables the simulation of various sets of rules
while altering the relevant allocation parameters. Moreover, this
package, in addition to being open source, requires less data to run
when compared to the Kidney-Pancreas Simulated Allocation Model (KPSAM)
software (\protect\hyperlink{ref-srtr}{Transplant Recipients 2015}) from
the Scientific Registry of Transplant Recipients, and as a result, it
may be used as a first method to create hypotheses that can then be
tested on KPSAM.

\hypertarget{kidney-allocation-systems}{%
\subsection{Kidney Allocation Systems}\label{kidney-allocation-systems}}

\hypertarget{portuguese-model}{%
\subsubsection{Portuguese Model}\label{portuguese-model}}

Portuguese rules on allocation of kidneys from deceased donor
(\textbf{PT model}) are based on a scoring system that takes in
consideration:\\
1. HLA mismatches between donor and transplant candidate; 1. Level of
immunization of the candidate; 1. Time on dialysis; 1. Age difference
between donor and transplant candidate (\texttt{pts()}).

Total scores for donor-recipient pairs are given by the column
\texttt{ptsPT}. Also, hipersensitized candidates (\texttt{hiper()}) (
calculated Panel Reactive Antibody \texttt{cPRA} \textgreater{} 85\%)
\texttt{HI} are prioritized and after that all candidates are ordered by
their corresponding score.

\hypertarget{euro-transplant-model}{%
\subsubsection{Euro Transplant Model}\label{euro-transplant-model}}

A simplified version of EuroTransplant Kidney Allocation System
(\textbf{ETKAS}) (\protect\hyperlink{ref-ET}{EuroTransplant 2020}) can
be simulated through \texttt{et()}. This applies to first time kidney
only candidates with more than 18 years old and that haven't donated one
of their own kidneys.

In this simulation for each donor, transplant candidates as sorted as :

\begin{enumerate}
\def\labelenumi{\arabic{enumi}.}
\tightlist
\item
  Senior Program (65+ years of old candidates when the donor has 65+
  years) \texttt{SP};
\item
  Acceptable Mismatch Program (candidates with a cPRA \textgreater{}
  85\% and without HLA antibodies against HLA's donor) \texttt{AM};
\item
  000 HLA mismatches (candidates without HLA mismatches with the donor)
  \texttt{mmHLA};
\item
  ETKAS points \texttt{pointsET}.
\end{enumerate}

Final points for each eligible candidate are obtained from the sum of
HLA points (\texttt{et\_mmHLA()}), dialysis (\texttt{et\_dialysis()})
points and MMP points (\texttt{et\_mmp()}).

\hypertarget{united-kingdom-model}{%
\subsubsection{United Kingdom Model}\label{united-kingdom-model}}

United Kingdom (\textbf{UK model}) deceased donor kidney allocation for
transplantation (\protect\hyperlink{ref-UK}{Blood and Transplant 2017})
(\texttt{uk()}) is firstly based on the definition of two ranked Tiers
of candidates eligible for the donor (\texttt{Tier}):

\begin{enumerate}
\def\labelenumi{\arabic{enumi}.}
\tightlist
\item
  Tier A -- patients with match score = 10 or \texttt{cPRA} = 100\% or
  time on \texttt{dialysis} \textgreater{} 7 years;
\item
  Tier B -- all other eligible patients.
\end{enumerate}

Within Tier A, transplant candidates are ordered by matchability and
time on dialysis. Transplant candidates within Tier B are prioritized
according to a point-based system computed with 7 elements:

\begin{enumerate}
\def\labelenumi{\arabic{enumi}.}
\tightlist
\item
  Matchability (\texttt{matchability})
\item
  Time on dialysis (\texttt{dialysis})
\item
  Donor-recipient risk index combinations (\texttt{ric()})
\item
  HLA match and age combined (\texttt{age} and \texttt{mmHLA})
\item
  Donor-recipient age difference (\texttt{age} and \texttt{donor\_age})
\item
  Total HLA mismatch (\texttt{mmHLa})
\item
  Blood group match (\texttt{abo\_uk()})
\end{enumerate}

This function simulates the allocation of kidneys to a candidates'
waiting list for kidney-only transplants and do not take in
consideration geographical criteria.

\hypertarget{limas-model}{%
\subsubsection{Lima's Model}\label{limas-model}}

And lastly, within \textbf{Lima's model}, a color prioritization
(\texttt{cp()}) of all waiting list transplant candidates is
established.

Transplant candidates are classified according to their clinical urgency
(red color), and regarding their time on \texttt{dialysis} and cPRA
value \texttt{cPRA}. With an orange color are marked those patients with
a \texttt{cPRA} \textgreater{} 85\% or with a time on \texttt{dialysis}
higher than waiting time 3rd quartile. As yellow are classified the
patients with a \texttt{cPRA} \textgreater{} 50\% or with a time on
\texttt{dialysis} higher than waiting time median. And, as green are
classified all the rest.

Within each color group candidates are ordered by \texttt{mmHLA}
(ascendant) and time on \texttt{dialisys} (descendant).

Also, candidates are allocated to donors within the same age group (old
to old program) (\texttt{SP}), mimicking EuroTransplant senior program
(\texttt{sp()}).

\hypertarget{candidates-selection-for-a-pool-of-donors}{%
\subsection{Candidates' selection for a pool of
donors}\label{candidates-selection-for-a-pool-of-donors}}

We can also simulate the selection of wait list candidates for a pool of
donors, according to a given model (or algorithm). The function
\texttt{donor\_recipient\_pairs()} allow us to compute all possible
donor-recipient pairs according to any of the previously described
kidney allocation algorithms.

An example where we use a pool of 70 donors (data frame available from
the package) to select from a wait list of 500 transplant candidates
(data frame also available from the package) is described by
\href{https://txopen.github.io/histoc/articles/cand_select.html}{candidates'
selection} vignette.

Also a new column for the estimated 5-year event (mortality or graft
failure) probability as described by
\protect\hyperlink{ref-Molnar:2018}{Molnar et al.}
(\protect\hyperlink{ref-Molnar:2018}{2018}){]} and available from the
application \href{https://balima.shinyapps.io/scoreTx/}{TxScore}, with
the function \texttt{txscore()}, can be computed.

\hypertarget{input-data}{%
\subsection{Input data}\label{input-data}}

Input data used on this package's functions, regarding either candidates
or donors information, when provided by the user must have the exact
same format as the example data available. Furthermore,
\href{https://github.com/txopen/simK}{\{simK\}}
(\protect\hyperlink{ref-simK}{Lima 2022}) package allows to generate
synthetic data both for candidates and donors that can be used with
\{histoc\}.

\hypertarget{bug-reports-and-contributions}{%
\subsection{Bug reports and
contributions}\label{bug-reports-and-contributions}}

Reports on any bug encountered or any request for a new feature, are
welcomed with a minimal reproducible example of your R code through the
link package website
\href{https://github.com/txopen/histoc/issues}{report a bug}, available
at the package website (\protect\hyperlink{ref-histoc}{Lima and Reis
2022})

\hypertarget{funding}{%
\section{Funding}\label{funding}}

This project received the ``Antonio Morais Sarmento'' research grant
from the Portuguese Society of Transplantation. This funding had no role
in: study design; software development; the writing of the report;
neither in the decision to submit the article for publication.

\hypertarget{references}{%
\section*{References}\label{references}}
\addcontentsline{toc}{section}{References}

\hypertarget{refs}{}
\begin{CSLReferences}{1}{0}
\leavevmode\vadjust pre{\hypertarget{ref-transplantr}{}}%
Asher, J. 2020. {``Transplantr: Audit and Research Functions for
Transplantation.''} \url{https://transplantr.txtools.net/}.

\leavevmode\vadjust pre{\hypertarget{ref-UK}{}}%
Blood, National Health Service, and Transplant. 2017. {``{UK Transplant
Registry, Organ Donation and Transplantation}.''}
\url{http://odt.nhs.uk/uk-transplant-registry/data/}.

\leavevmode\vadjust pre{\hypertarget{ref-data.table}{}}%
Dowle, M., and A. Srinivasan. 2021. {``Data.table: Extension of
`Data.frame`.''} \url{https://CRAN.R-project.org/package=data.table}.

\leavevmode\vadjust pre{\hypertarget{ref-ET}{}}%
EuroTransplant. 2020. {``{ETKAS and ESP}.''}
\url{https://www.eurotransplant.org/allocation/eurotransplant-manual/}.

\leavevmode\vadjust pre{\hypertarget{ref-Geddes:2005}{}}%
Geddes, C., R. Rodger, C. Smith, and A. Ganai. 2005. {``{Allocation of
deceased donor kidneys for transplantation: Opinions of patients with
CKD}.''} \emph{American Journal of Kidney Diseases} 46 (5): 949--56.
\url{https://doi.org/10.1053/j.ajkd.2005.07.031}.

\leavevmode\vadjust pre{\hypertarget{ref-simK}{}}%
Lima, B. 2022. {``simK: Synthetic Data on Kidney Transplantation.''}
Transplant Open Registry. \url{https://txopen.github.io/simK/}.

\leavevmode\vadjust pre{\hypertarget{ref-Lima:2020}{}}%
Lima, B., and H. Alves. 2020. {``{Access to kidney transplantation: a
bioethical perspective}.''} \emph{Portuguese Journal of Nephrology \&
Hypertension} 34 (2): 76--78.
\url{https://doi.org/10.32932/pjnh.2020.07.070}.

\leavevmode\vadjust pre{\hypertarget{ref-Lima:2013}{}}%
Lima, B., M. Mendes, and H. Alves. 2013. {``{Kidney Transplant
allocation in Portugal}.''} \emph{Portuguese Journal of Nephrology \&
Hypertension} 27 (4): 313--16.
\url{http://www.bbg01.com/cdn/clientes/spnefro/pjnh/46/artigo_14.pdf}.

\leavevmode\vadjust pre{\hypertarget{ref-histoc}{}}%
Lima, B., and F. Reis. 2022. \emph{Histoc: Histocompatibility Functions
for Organ Transplantation}. Transplant Open Registry.
\url{https://txopen.github.io/histoc/}.

\leavevmode\vadjust pre{\hypertarget{ref-Molnar:2018}{}}%
Molnar, M., D. Nguyen, Y. Chen, V. Ravel, E. Streja, M. Krishnan, C.
Kovesdy, and K. Kalantar-zadeh. 2018. {``{Predictive Score for
Posttransplantation Outcomes}.''} \emph{Transplantation} 101 (6):
1353--64. \url{https://doi.org/10.1097/TP.0000000000001326}.

\leavevmode\vadjust pre{\hypertarget{ref-R}{}}%
R Core Team. 2021. {``R: A Language and Environment for Statistical
Computing.''} Vienna, Austria: R Foundation for Statistical Computing.
\url{https://www.R-project.org/}.

\leavevmode\vadjust pre{\hypertarget{ref-srtr}{}}%
Transplant Recipients, Scientific Registry of. 2015. {``Kidney-Pancreas
Simulated Allocation Model (KPSAM).''}
\url{https://srtr.org/requesting-srtr-data/simulated-allocation-models/}.

\end{CSLReferences}

\end{document}
